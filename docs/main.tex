\documentclass{article}
\usepackage{graphics}
\usepackage[utf8]{inputenc}
\usepackage[italian]{babel}
\usepackage[document]{ragged2e}
\usepackage{graphicx}
\usepackage{float}
\graphicspath{ {./images/} }

\begin{document}
\pagenumbering{arabic}
\renewcommand{\labelenumii}{\arabic{enumi}.\arabic{enumii}}
\begin{titlepage}
   \begin{center}
        \Huge        
        \textbf{Documentazione Progetto Ingegneria del Software}
            
        \vspace{1.5cm}
        \Large
        \textbf{Pietro Dudine, VR456578}
       
        \textbf{Davide Rossignolo, VR456503}

        \vfill
            
        \vspace{0.8cm}
            
        Università di Verona
        
        Giugno 2022
            
   \end{center}
\end{titlepage}

\tableofcontents
\newpage

\section{Requisiti ed interazioni utente-sistema}
    \subsection{Specifiche dei casi d'uso}
    \textbf{Descrizione generale}
    
    Il sistema proposto verrà utilizzato dall'agenzia che gestisce i lavoratori stagionali. Il personale la prima volta che accede al sistema dovrà fornire i propri dati anagrafici per registrarsi come utente autorizzato. Una volta ottenute le credenziali proseguirà con l'autenticazione e se questa è andata a buon fine potrà iniziare ad utilizzare la piattaforma per il proprio lavoro.
    
    Ogni dipendente autenticato potrà eseguire le seguenti operazioni:
    \begin{itemize}
        \item Inserire i dati del lavoratore
        \item Ricercare dei lavoratori 
        \item Visualizzare i dati
        \item Modificare i dati anagrafici o le esperienze lavorative
    \end{itemize}
    
    \begin{figure}[H]
        \includegraphics[width=\linewidth]{use_case.png}
        \caption{Diagramma dei casi d'uso.}
        \label{fig:use_case}
    \end{figure}
    
    \subsubsection{Caso d'uso login}
    \subsubsection{Caso d'uso inserimento dati dei lavoratori}
    \subsubsection{Caso d'uso modifica dei dati}
    
\end{document}
